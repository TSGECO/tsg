\documentclass[11pt,a4paper]{article}

\usepackage[utf8]{inputenc}
\usepackage[T1]{fontenc}
\usepackage{lmodern}
\usepackage{microtype}
\usepackage{amsmath}
\usepackage{amssymb}
\usepackage{graphicx}
\usepackage{hyperref}
\usepackage{xcolor}
\usepackage{booktabs}
\usepackage{enumitem}
\usepackage[dvipsnames]{xcolor}
\usepackage{geometry}
\usepackage{fancyhdr}

% Minted package for beautiful syntax highlighting
\usepackage{minted}
\usemintedstyle{borland}
\setminted{
  fontsize=\small,
  breaklines=true,
  autogobble,
  frame=single,
  framesep=2mm,
  linenos
}

% Use bash lexer for TSG code examples (since it handles # comments well)
\newminted{bash}{
  fontsize=\small,
  breaklines=true,
  autogobble,
  frame=single,
  framesep=2mm,
  linenos
}

% Define a standard environment for TSG code (using \newenvironment instead of \renewenvironment)
\newenvironment{tsgcode}
  {\VerbatimEnvironment\begin{minted}[fontsize=\small,breaklines=true,autogobble,frame=single,framesep=2mm,linenos]{bash}}
  {\end{minted}}

% Document setup
\geometry{a4paper, margin=1in}
\hypersetup{
    colorlinks=true,
    linkcolor=NavyBlue,
    filecolor=MidnightBlue,
    urlcolor=RoyalBlue,
    citecolor=OliveGreen,
    pdftitle={Transcript Segment Graph (TSG) Format Specification},
    pdfauthor={TSGECO},
}

% Header and footer setup
\pagestyle{fancy}
\fancyhf{}
\fancyhead[L]{TSG Format Specification}
\fancyhead[R]{\thepage}
\fancyfoot[C]{TSGECO \copyright \the\year}

\title{\textbf{Transcript Segment Graph (TSG)\\Format Specification}}
\author{Yangyang Li}
\date{\today}

\begin{document}

\maketitle

\begin{abstract}
	The Transcript Segment Graph (TSG) format provides a comprehensive, graph-based representation for transcript assemblies, splicing relationships, and transcript variants. Based on concepts from the Graphical Fragment Assembly (GFA) 2.0 specification, TSG allows researchers to represent exons, splice junctions, isoforms, and structural variants within a single file format. This document defines the complete TSG specification, including record types, semantics, and implementation guidelines.
\end{abstract}

\tableofcontents
\newpage

\section{Overview}

The Transcript Segment Graph (TSG) format is designed for representing transcript assemblies and splicing relationships, adapting concepts from the Graphical Fragment Assembly (GFA) 2.0 specification.
TSG allows for the representation of exons, splice junctions, isoforms, and structural variants in a graph-based format. The format supports multiple graphs within a single file.

\section{Conceptual Model}

In the TSG model:

\begin{enumerate}[leftmargin=*]
	\item \textbf{Graphs (G)} represent independent transcript graphs, each with its own set of nodes and edges.
	\item \textbf{Chains (C)} are used to build the graph structure. They define the nodes and edges that make up the graph.
	\item \textbf{Paths (P)} are traversals through the constructed graph.
	\item The complete TSG is built by combining all nodes and edges from all chains within each graph.
	\item After constructing the graph from chains, paths can be defined to represent ways of traversing the graph.
\end{enumerate}

\section{File Format}

TSG is a tab/space-delimited text format with each line beginning with a single letter that defines the record type.
Fields within each line are separated by tabs/spaces, and each line represents a distinct element of the transcript graph.

\section{Multi-Graph Support}

TSG supports multiple graphs within a single file using:

\begin{enumerate}[leftmargin=*]
	\item \textbf{Graph separators (G)} to define the start of each graph section
	\item \textbf{Graph ID namespace} to qualify element IDs with their respective graph
\end{enumerate}

All elements in a graph section belong to that graph until a new graph section is encountered.

\section{Record Types}

\subsection{Global Header (H)}

Contains metadata about the entire file.
These appear at the beginning of the file before any graph sections.

\begin{tsgcode}
	H  <tag>  <value>
\end{tsgcode}

Fields:
\begin{itemize}[leftmargin=*]
	\item \texttt{tag}: Identifier for the header entry
	\item \texttt{value}: Header value
\end{itemize}

\subsection{Graph Separator (G)}

Indicates the start of a new graph section and provides graph metadata.

\begin{tsgcode}
	G  <graph_id>  [<tag>:<type>:<value> ...]
\end{tsgcode}

Fields:
\begin{itemize}[leftmargin=*]
	\item \texttt{graph\_id}: Unique identifier for the graph
	\item Optional list of attribute tags in \texttt{tag:type:value} format
\end{itemize}

Example:
\begin{tsgcode}
	G  gene_a  name:Z:BRCA1  source:Z:RefSeq  version:Z:GRCh38
\end{tsgcode}

\subsection{Nodes (N)}

Represent exons or transcript segments.

\begin{tsgcode}
	N  <id>  <genomic_location>  <reads>  [<seq>]
\end{tsgcode}

Fields:
\begin{itemize}[leftmargin=*]
	\item \texttt{id}: Unique identifier for the node (within the current graph section)
	\item \texttt{genomic\_location}: Format \texttt{chromosome:strand:coordinates} where:
	      \begin{itemize}
		      \item \texttt{chromosome}: Chromosome name (e.g., "chr1")
		      \item \texttt{strand}: "+" for forward strand, "-" for reverse strand
		      \item \texttt{coordinates}: Comma-separated list of exon coordinates in "start-end" format
	      \end{itemize}
	\item \texttt{reads}: Comma-separated list of reads supporting this node, in format \texttt{read\_id:type}
	      \begin{itemize}
		      \item Types might include SO (spanning), IN (internal), SI (significant), etc.
	      \end{itemize}
	\item \texttt{seq} (optional): Sequence of the node
\end{itemize}

\subsection{Edges (E)}

Represent connections between nodes, including splice junctions or structural variants.

\begin{tsgcode}
	E  <id>  <source_id>  <sink_id>  <SV>
\end{tsgcode}

Fields:
\begin{itemize}[leftmargin=*]
	\item \texttt{id}: Unique identifier for the edge (within the current graph section)
	\item \texttt{source\_id}: ID of the source node
	\item \texttt{sink\_id}: ID of the target node
	\item \texttt{SV}: Structural variant information in format "reference\_name1,reference\_name2,breakpoint1,breakpoint2,sv\_type"
\end{itemize}

\subsection{Unordered Groups/Sets (U)}

Represent unordered collections of graph elements.

\begin{tsgcode}
	U  <group_id>  <element_id_1> <element_id_2> ... <element_id_n>
\end{tsgcode}

Fields:
\begin{itemize}[leftmargin=*]
	\item \texttt{group\_id}: Unique identifier for the unordered group (within the current graph section)
	\item \texttt{element\_id\_*}: Space-separated list of element identifiers (nodes, edges, or other groups)
\end{itemize}

\subsection{Ordered Groups/Paths (P)}

Represent ordered collections of graph elements where orientation matters.

\begin{tsgcode}
	P  <path_id>  <oriented_element_id_1> <oriented_element_id_2> ... <oriented_element_id_n>
\end{tsgcode}

Fields:
\begin{itemize}[leftmargin=*]
	\item \texttt{path\_id}: Unique identifier for the ordered group (within the current graph section)
	\item \texttt{oriented\_element\_id\_*}: Space-separated list of element identifiers with orientation (+ or -)
\end{itemize}

\subsection{Chains (C)}

Represent explicit paths through the graph with alternating nodes and edges.

\begin{tsgcode}
	C  <chain_id>  <node_id_1> <edge_id_1> <node_id_2> <edge_id_2> ... <node_id_n>
\end{tsgcode}

Fields:
\begin{itemize}[leftmargin=*]
	\item \texttt{chain\_id}: Unique identifier for the chain (within the current graph section)
	\item Elements: Space-separated list of alternating node and edge identifiers
	      \begin{itemize}
		      \item Must start and end with node identifiers
		      \item Must have an odd number of elements
		      \item Adjacent elements must be connected in the graph
	      \end{itemize}
\end{itemize}

\subsection{Attributes (A)}

Optional metadata attached to other elements.

\begin{tsgcode}
	A  <element_type>  <element_id>  <tag>:<type>:<value>
\end{tsgcode}

Fields:
\begin{itemize}[leftmargin=*]
	\item \texttt{element\_type}: Type of element (N, E, U, P, or C)
	\item \texttt{element\_id}: Identifier of the element to attach the attribute to
	\item \texttt{tag}: Name of the attribute
	\item \texttt{type}: Single letter code for the attribute data type
	\item \texttt{value}: Value of the attribute
\end{itemize}

\subsection{Inter-Graph Links (L)}

Represents connections between elements in different graphs.

\begin{tsgcode}
	L  <id>  <graph_id1>:<element_id1>  <graph_id2>:<element_id2>  <link_type>  [<tag>:<type>:<value> ...]
\end{tsgcode}

Fields:
\begin{itemize}[leftmargin=*]
	\item \texttt{id}: Unique identifier for the link
	\item \texttt{graph\_id1:element\_id1}: Graph-qualified identifier for the first element
	\item \texttt{graph\_id2:element\_id2}: Graph-qualified identifier for the second element
	\item \texttt{link\_type}: Type of link (e.g., "fusion", "reference", "containment")
	\item Optional list of attribute tags in \texttt{tag:type:value} format
\end{itemize}

\section{Semantics}

\subsection{Graph Sections}

Each graph section in a TSG file defines an independent transcript segment graph:

\begin{enumerate}[leftmargin=*]
	\item \textbf{Section Boundaries}: A graph section begins with a G record and continues until the next G record or the end of the file.
	\item \textbf{Element Scope}: All elements (N, E, C, P, U, A) defined within a graph section belong to that graph.
	\item \textbf{Element Reference}: Elements can only reference other elements within the same graph section, except through inter-graph links (L).
	\item \textbf{Element IDs}: Element IDs must be unique within their graph section but can be reused in different graph sections.
\end{enumerate}

\subsection{Node Semantics}

Nodes in TSG represent exons or transcript segments.
Each node has a genomic location that includes chromosome, strand, and coordinates.
The genomic location specifies where the node is located in the reference genome.
Nodes can be supported by different types of read evidence through the \texttt{reads} field, and can optionally include the sequence.

\subsection{Edge Semantics}

Edges in TSG represent connections between nodes, such as splice junctions or structural variants.
The \texttt{SV} field provides details about the genomic context of the connection, including reference names, breakpoints, and the type of structural variant or splice.

\subsection{Read Continuity}

Read continuity is a critical concept in TSG that ensures valid traversals through the graph:

\begin{enumerate}[leftmargin=*]
	\item \textbf{Definition}: Read continuity requires that specific patterns of read support exist between connected nodes in a path, depending on node types.

	\item \textbf{Node Types and Continuity Requirements}:
	      \begin{itemize}
		      \item \textbf{SO (Source Node)}: Represents the start of a read. No read continuity required with previous nodes.
		      \item \textbf{SI (Sink Node)}: Represents the end of a read. No read continuity required with subsequent nodes.
		      \item \textbf{IN (Intermediary Node)}: Represents an internal segment of a read. Must share at least one read ID with both its previous and next nodes in the path.
	      \end{itemize}

	\item \textbf{Validation}:
	      \begin{itemize}
		      \item For IN nodes, implementations must verify that at least one common read ID exists between the current node and both its adjacent nodes.
		      \item SO and SI nodes provide more flexible continuity, allowing for extended paths without requiring end-to-end read support.
		      \item A path can be considered valid even if it doesn't have a single read spanning its entire length.
	      \end{itemize}

	\item \textbf{Constraints}:
	      \begin{itemize}
		      \item Each IN node in a valid path must maintain read continuity with its adjacent nodes.
		      \item The specific read type (SO, SI, IN) determines the continuity requirements at each position in the path.
	      \end{itemize}

	\item \textbf{Representation}: The read IDs in each node's \texttt{reads} field implicitly define the continuity relationships in the graph, while the read types determine the continuity constraints.
\end{enumerate}

\subsection{Chains vs. Paths}

Chains and paths serve fundamentally different purposes in the TSG format:

\begin{enumerate}[leftmargin=*]
	\item \textbf{Chains as Graph Construction Elements}:
	      \begin{itemize}
		      \item Chains (C) are used to build the TSG graph itself.
		      \item Each chain contributes nodes and edges to the graph structure.
		      \item The complete graph is constructed by collecting all nodes and edges from all chains.
		      \item Chains represent the source evidence (e.g., transcript sequences) from which the graph was built.
	      \end{itemize}

	\item \textbf{Paths as Graph Traversals}:
	      \begin{itemize}
		      \item Paths (P) are traversals through the already-constructed graph.
		      \item Paths don't add any new structural elements to the graph.
		      \item They represent ways of traveling through the existing nodes and edges.
		      \item Paths can represent transcript isoforms, alternative splicing patterns, or other biological features.
	      \end{itemize}
\end{enumerate}

This distinction is critical: chains define what the graph is, while paths define ways to traverse the graph.

\subsection{Inter-Graph Links}

Inter-graph links provide a way to represent relationships between elements in different graphs:

\begin{enumerate}[leftmargin=*]
	\item \textbf{Types of Links}:
	      \begin{itemize}
		      \item \textbf{Fusion}: Represents a fusion between transcripts in different graphs
		      \item \textbf{Reference}: Indicates that one element references another
		      \item \textbf{Containment}: Shows that one element contains or is a superset of another
		      \item \textbf{Identity}: Indicates that elements across graphs are identical
	      \end{itemize}

	\item \textbf{Usage Scenarios}:
	      \begin{itemize}
		      \item Connecting fusion transcripts across genes
		      \item Linking alternative assemblies of the same region
		      \item Creating hierarchical relationships between graphs
		      \item Cross-referencing between different transcript annotations
	      \end{itemize}
\end{enumerate}

\section{Processing Model}

The typical processing flow for a TSG file depends on whether the graph structure (nodes and edges) already exists:

\subsection{Case 1: Nodes and Edges Are Explicitly Defined}

If the TSG file contains explicit node (N) and edge (E) records:

\begin{enumerate}[leftmargin=*]
	\item Process the global headers at the beginning of the file
	\item For each graph section (started by a G record):
	      \begin{itemize}
		      \item Create a new graph with the given ID and attributes
		      \item Read and create the graph structure directly from the records in the section
		      \item Process chains (C) as additional evidence or source information
		      \item Process paths (P) as traversals through the explicitly defined graph
	      \end{itemize}
	\item Process inter-graph links (L) to establish connections between graphs
	\item Validate read continuity by checking for shared read IDs across adjacent nodes in paths
\end{enumerate}

\subsection{Case 2: Nodes and Edges Are Not Explicitly Defined}

If the TSG file does not contain explicit node and edge records, or contains only partial definitions:

\begin{enumerate}[leftmargin=*]
	\item Process the global headers at the beginning of the file
	\item For each graph section (started by a G record):
	      \begin{itemize}
		      \item Create a new graph with the given ID and attributes
		      \item Extract and construct all nodes and edges from chains (C) in the section
		      \item Build the complete graph structure from these extracted elements
		      \item Process paths (P) as traversals through the graph constructed from chains
	      \end{itemize}
	\item Process inter-graph links (L) to establish connections between graphs
	\item Verify read continuity for all paths by ensuring adjacent nodes share common read support
\end{enumerate}

\section{Type Definitions for Attributes}

\begin{itemize}[leftmargin=*]
	\item \texttt{i}: Integer
	\item \texttt{f}: Float
	\item \texttt{Z}: String
	\item \texttt{J}: JSON
	\item \texttt{H}: Hex string
	\item \texttt{B}: Byte array
\end{itemize}

\section{Example}

\begin{tsgcode}
	# Global headers
	H  TSG  1.0
	H  reference  GRCh38

	# First graph
	G  gene_a  name:Z:BRCA1  locus:Z:chr17q21.31

	# Nodes for gene_a
	N  n1  chr17:+:41196312-41196402  read1:SO,read2:SO  ACGTACGT
	N  n2  chr17:+:41199660-41199720  read2:IN,read3:IN  TGCATGCA
	N  n3  chr17:+:41203080-41203134  read1:SI,read2:SI  CTGACTGA

	# Edges for gene_a
	E  e1  n1  n2  chr17,chr17,41196402,41199660,DEL
	E  e2  n2  n3  chr17,chr17,41199720,41203080,INV

	# Chains for gene_a
	C  chain1  n1  e1  n2  e2  n3

	# Paths for gene_a
	P  transcript1  n1+  e1+  n2+  e2+  n3+

	# Sets for gene_a
	U  exon_set  n1  n2  n3

	# Attributes for gene_a elements
	A  N  n1  expression:f:10.5
	A  N  n1  ptc:i:10
	A  P  transcript1  tpm:f:8.2

	# Second graph
	G  gene_b  name:Z:BRCA2  locus:Z:chr13q13.1

	# Nodes for gene_b
	N  n1  chr13:+:32315480-32315652  read4:SO,read5:SO  GATTACA
	N  n2  chr13:+:32316528-32316800  read4:IN,read5:IN  TACGATCG
	N  n3  chr13:+:32319077-32319325  read4:SI,read5:SI  CGTACGTA

	# Edges for gene_b
	E  e1  n1  n2  chr13,chr13,32315652,32316528,splice
	E  e2  n2  n3  chr13,chr13,32316800,32319077,splice

	# Chains for gene_b
	C  chain1  n1  e1  n2  e2  n3

	# Paths for gene_b
	P  transcript1  n1+  e1+  n2+  e2+  n3+

	# Sets for gene_b
	U  exon_set  n1  n2  n3

	# Attributes for gene_b elements
	A  P  transcript1  tpm:f:3.7

	# Inter-graph links (appears after all graph sections)
	L  fusion1  gene_a:n3  gene_b:n1  fusion  type:Z:chromosomal
\end{tsgcode}

In this example:

\begin{enumerate}[leftmargin=*]
	\item The file begins with global headers that apply to the entire file
	\item Two graph sections are defined (gene\_a and gene\_b), each started by a G record
	\item Each graph has its own nodes, edges, chains, paths, and attributes
	\item Element IDs (n1, e1, etc.) are local to each graph section
	\item An inter-graph link represents a fusion between elements from different graphs
	\item To reference elements across graphs (in the L record), graph-qualified IDs are used (graph\_id:element\_id)
\end{enumerate}

\section{Implementation Considerations}

\subsection{Graph Section Handling}

Implementations should:

\begin{itemize}[leftmargin=*]
	\item Initialize a new graph context whenever a G record is encountered
	\item Associate all subsequent elements with the current graph until the next G record
	\item Maintain separate data structures for each graph
	\item Enforce that connections (edges, chains, paths) only exist between elements in the same graph
	\item Process inter-graph links separately from within-graph connections
\end{itemize}

\subsection{Element ID Resolution}

When processing elements:

\begin{itemize}[leftmargin=*]
	\item Within a graph section, element IDs are resolved in the context of that graph
	\item In inter-graph links, element IDs must be qualified with their graph ID (graph\_id:element\_id)
	\item Implementations should maintain a mapping of (graph\_id, element\_id) pairs to resolve references
\end{itemize}

\subsection{Genomic Location Parsing}

Implementations should carefully parse the genomic location field, which contains:

\begin{itemize}[leftmargin=*]
	\item Chromosome (e.g., "chr1")
	\item Strand ("+" or "-")
	\item Coordinates (comma-separated list of "start-end" pairs)
\end{itemize}

These components are separated by colons.

\subsection{Read Evidence}

Read evidence is recorded with both read identifiers and types. Implementations should:

\begin{itemize}[leftmargin=*]
	\item Parse the read identifier and read type, separated by a colon
	\item Support different read types (SO, IN, SI, etc.) as used in the implementation
\end{itemize}

\subsection{Validation Requirements}

Implementations should validate that:

\begin{itemize}[leftmargin=*]
	\item Each graph section has a unique graph ID
	\item Element IDs are unique within their graph section
	\item Chains have an odd number of elements (starting and ending with nodes)
	\item Adjacent elements in chains are correctly connected in the graph
	\item Group identifiers are unique across U, P, and C types within each graph section
	\item Paths (P-lines) only reference elements that exist in the same graph section
	\item Inter-graph links only reference elements that exist in their respective graphs
\end{itemize}

\subsection{Chain Processing}

\begin{itemize}[leftmargin=*]
	\item Process chains within each graph section independently
	\item When encountering a chain, extract all nodes and edges and add them to the current graph
	\item The same node or edge can appear in multiple chains within the same graph
	\item The structural integrity of each graph is defined by its chains
\end{itemize}

\subsection{Path Processing}

\begin{itemize}[leftmargin=*]
	\item Paths do not add new elements to the graph
	\item Paths must reference existing nodes and edges within the same graph section
	\item Paths can include orientation information (+ or -) for elements
\end{itemize}

\subsection{Read Continuity Verification}

When processing TSG files, implementations should:

\begin{itemize}[leftmargin=*]
	\item Extract read IDs from each node's \texttt{reads} field along with their types (SO, SI, IN)
	\item For each path or chain traversal:
	      \begin{itemize}
		      \item For IN nodes: Ensure at least one read ID is shared with both the previous and next nodes
		      \item For SO nodes: No continuity check required with previous nodes
		      \item For SI nodes: No continuity check required with subsequent nodes
	      \end{itemize}
	\item Flag paths where IN nodes lack proper read continuity as potentially unsupported by the data
	\item Recognize that valid paths may be constructed even without end-to-end read support, as long as the IN node continuity requirements are satisfied
	\item Provide options to filter or annotate paths based on different levels of read continuity stringency
\end{itemize}

\subsection{Biological Interpretation in Transcript Analysis}

In the context of transcript analysis, the TSG format elements typically represent:

\subsubsection{Graphs (G)}

\begin{itemize}[leftmargin=*]
	\item \textbf{Genes}: Independent genetic loci
	\item \textbf{Transcription Units}: Coordinated transcriptional regions
	\item \textbf{Alternative Assemblies}: Different representations of the same region
\end{itemize}

\subsubsection{Nodes (N)}

\begin{itemize}[leftmargin=*]
	\item \textbf{Exons}: Genomic regions that are transcribed and remain in the mature RNA
	\item Can include multiple segments (e.g., for complex exon structures)
	\item Read support indicates which sequencing reads support this exon, with different types of support
\end{itemize}

\subsubsection{Edges (E)}

\begin{itemize}[leftmargin=*]
	\item \textbf{Splice Junctions}: Connections between exons
	\item \textbf{Structural Variants}: Genomic rearrangements like fusions, deletions, or insertions
	\item The SV field provides details on the exact genomic coordinates
\end{itemize}

\subsubsection{Chains (C)}

\begin{itemize}[leftmargin=*]
	\item \textbf{Original Transcripts}: Complete transcript sequences observed in the data
	\item \textbf{Source Evidence}: The actual RNA molecules detected
	\item \textbf{Assembled Transcripts}: Transcripts assembled from read data
	\item These build the structure of the transcript graph
\end{itemize}

\subsubsection{Paths (P)}

\begin{itemize}[leftmargin=*]
	\item \textbf{Transcript Isoforms}: Alternative splicing variants
	\item \textbf{Expression Patterns}: Different ways genes are expressed
	\item \textbf{Predicted Transcripts}: Computationally predicted transcript models
	\item These are ways to traverse the established transcript graph
\end{itemize}

\subsubsection{Inter-Graph Links (L)}

\begin{itemize}[leftmargin=*]
	\item \textbf{Fusion Transcripts}: Transcripts spanning multiple genes
	\item \textbf{Reference Relationships}: Cross-references between different transcript annotations
	\item \textbf{Containment Relationships}: Hierarchical organization of transcripts
\end{itemize}

This separation aligns with how many transcript assembly algorithms work:

\begin{enumerate}[leftmargin=*]
	\item First, chains of exons and splice junctions are identified from the data
	\item Then, potential transcripts are derived by traversing the graph in different ways
	\item Finally, relationships between different transcript graphs are established
\end{enumerate}

\end{document}
